%% start of file `template.tex'.
%% Copyright 2006-2013 Xavier Danaux (xdanaux@gmail.com).
%
% This work may be distributed and/or modified under the
% conditions of the LaTeX Project Public License version 1.3c,
% available at http://www.latex-project.org/lppl/.


\documentclass[11pt,a4paper,sans]{moderncv}        % possible options include font size ('10pt', '11pt' and '12pt'), paper size ('a4paper', 'letterpaper', 'a5paper', 'legalpaper', 'executivepaper' and 'landscape') and font family ('sans' and 'roman')

% moderncv themes
\moderncvstyle{classic}                             % style options are 'casual' (default), 'classic', 'oldstyle' and 'banking'
\moderncvcolor{blue}                               % color options 'blue' (default), 'orange', 'green', 'red', 'purple', 'grey' and 'black'
%\renewcommand{\familydefault}{\sfdefault}         % to set the default font; use '\sfdefault' for the default sans serif font, '\rmdefault' for the default roman one, or any tex font name
%\nopagenumbers{}                                  % uncomment to suppress automatic page numbering for CVs longer than one page

% character encoding
\usepackage[utf8]{inputenc}                       % if you are not using xelatex ou lualatex, replace by the encoding you are using
%\usepackage{CJKutf8}                              % if you need to use CJK to typeset your resume in Chinese, Japanese or Korean
\usepackage{hyperref}
\hypersetup{
    colorlinks=true,
    linkcolor=blue,
    filecolor=magenta,      
    urlcolor=blue,
}
% \usepackage[hidelinks]{hyperref}x
% \usepackage{xcolor,soul,lipsum}
% \newcommand{\myul}[2][black]{\setulcolor{#1}\ul{#2}\setulcolor{black}}


% adjust the page margins
\usepackage[scale=0.78, top=0.5in, bottom=0.5in]{geometry}
%\setlength{\hintscolumnwidth}{3cm}                % if you want to change the width of the column with the dates
%\setlength{\makecvtitlenamewidth}{10cm}           % for the 'classic' style, if you want to force the width allocated to your name and avoid line breaks. be careful though, the length is normally calculated to avoid any overlap with your personal info; use this at your own typographical risks...

% personal data
\vspace{-2cm}
\name{Hardik}{Shah}
% \title{Resumé title}                               % optional, remove / comment the line if not wanted
% \address{Rue de la Blancherie 17}{Chavannes-près-Renens 1022}{Lausanne}% optional, remove / comment the line if not wanted; the "postcode city" and and "country" arguments can be omitted or provided empty
% \phone[mobile]{+91 94 279 25 103}                   % optional, remove / comment the line if not wanted
% \phone[fixed]{+2~(345)~678~901}                    % optional, remove / comment the line if not wanted
% \phone[fixed]{(+91) 9427925103}                      % optional, remove / comment the line if not wanted
\email{hardik01shah@gmail.com}                               % optional, remove / comment the line if not wanted
\homepage{hardik01shah.github.io}                         % optional, remove / comment the line if not wanted
\phone[mobile]{www.linkedin.com/in/hardik01shah}                         % optional, remove / comment the line if not wanted
% \extrainfo{additional information}                 % optional, remove / comment the line if not wanted
% \photo[64pt][0.4pt]{picture}                       % optional, remove / comment the line if not wanted; '64pt' is the height the picture must be resized to, 0.4pt is the thickness of the frame around it (put it to 0pt for no frame) and 'picture' is the name of the picture file
% \quote{Some quote}                                 % optional, remove / comment the line if not wanted

% to show numerical labels in the bibliography (default is to show no labels); only useful if you make citations in your resume
%\makeatletter
%\renewcommand*{\bibliographyitemlabel}{\@biblabel{\arabic{enumiv}}}
%\makeatother
%\renewcommand*{\bibliographyitemlabel}{[\arabic{enumiv}]}% CONSIDER REPLACING THE ABOVE BY THIS

% bibliography with mutiple entries
%\usepackage{multibib}
%\newcites{book,misc}{{Books},{Others}}
%----------------------------------------------------------------------------------
%            content
%----------------------------------------------------------------------------------
\nopagenumbers

\begin{document}
%\begin{CJK*}{UTF8}{gbsn}                          % to typeset your resume in Chinese using CJK
%-----       resume       ---------------------------------------------------------

\makecvtitle
\vspace{-1.2 cm}
\section{Education}
\cventry{2023--Present}{MSc in Computer Science}{ETH Zurich}{}{}
{}
{}
\cventry{2019--2023}{B.E. in Computer Science Engineering}{BITS Pilani, Goa}{}{\textit{9.64/10}}
{\textbf{w/ Minor in Data Science}}
{}
\cvlistitem{\textbf{Institute Rank 6} in a batch of 900 students}
\cvlistitem{Recipient of BITS Goa \textbf{Merit Scholarship} for all 8 semesters awarded to \textbf{top 10} students across all departments– 100\% tuition fee waiver.}
% \vspace{0. cm}
\normalsize

\section{Research Experience}
\cvitem{Institution}{\textbf{Google Research}
\hfill [Jan'23-Jun'23] 
\hspace{5 em}\newline
\textit{Student Researcher, Machine Learning and Optimization Team} 
% \newline
% \scriptsize{Under review at ICLR '24}
}
\cvitem{Project Title}{End-to-End Neural Network Compression via $\frac{l_1}{l_2}$ latency surrogates  [\href{https://arxiv.org/abs/2306.05785}{preprint*}]}
% \cvitem{}{}
\cvitem{Description}
{Developed a versatile \textbf{neural network compression} toolbox that optimizes for the model's FLOPs via a novel $\frac{l_1}{l_2}$ latency surrogate in various compression methods, including \textbf{pruning} and \textbf{low-rank factorization}. Extended the FLOPs regularizer to optimize over actual on-device latency using a latency look-up table of the target device. Achieved 11\% reduction in latency on Pixel-6, and 15\% reduction in FLOPs in compressing MobileNetV3 on ImageNet-1K without drop in accuracy, while still requiring 3× less training compute than SOTA NAS techniques. For BERT compression on GLUE fine-tuning tasks, we achieve 50\% reduction in FLOPs with only 1\% drop in performance. 
\newline
\scriptsize{\textit{*Under review at ICLR '24}}
}

\vspace{0.3 cm}

\cvitem{Institution}{\textbf{Google Research}
\hfill [Aug'22-Dec'22] 
\hspace{5 em}\newline
\textit{Student Researcher, Machine Learning and Optimization Team} \newline
\scriptsize{\textit{Undergraduate Thesis, Supervisor:  \textbf{\href{https://www.prateekjain.org/}{Dr. Prateek Jain}} (Sr. Staff Research Scientist, Google)}}
}
\cvitem{Project Title}{Machine Learning Optimization for object detection on low-end smartphones.}
\cvitem{Description}
{Optimized on-device latency of large \textbf{character recognition models} used for OCR tasks in Google products like Lens, for \textbf{faster on-device inference} while maintaining accuracy. Improved parameter efficiency for OCR tasks by extending Singular Value Decomposition(SVD) techniques and Orthogonal Matching Pursuit(\href{https://ieeexplore.ieee.org/abstract/document/342465/}{OMP}) on 1x1 convolution kernels. Experimentally observed constant performance with 33\% less parameters and 10\% reduction in latency. Additionally reduced on-device latency of \textbf{MobileNet} architectures for QR-code detection with GooglePay team.}

% \cvitem{Project Title}{Using Machine Learning and Optimization techniques for running complex object detection and vision algorithms on cheap smart phones}
% \cvitem{Description}
% {Applied optimizations on the GMLP and ConvNext architectures for SSD network for OCR tasks in Google products. SVD for reconstructing original convolution kernels with lesser parameters for reduced model size. Formation of an ILP problem for choosing optimal balance between rank/energy split for SVD. Experimentation with spatial smoothness architectures that involve using less intermediate features at the bottleneck and using interpolation to estimate remaining features. Reduce on-device latency of QR-code detection and decoding that use Mobilenetv1 and Mobilenetv3 architectures. Experimented with spatial smoothness architecture and Hyperparam tuning - depth multipliers, optimizers, number of blocks etc.}
\vspace{0.3 cm}

\cvitem{Institution}{\textbf{Robot Vision Lab, Karlsruhe University of Applied Sciences}
\hfill [May'22-Aug'22] 
\hspace{5 em}\newline
\textit{Summer Research Intern, DAAD WISE Scholarship} \newline
\scriptsize{\textit{Supervisor:  \textbf{\href{https://www.niclas-zeller.de/}{Prof. Niclas Zeller}}}}
}
\cvitem{Project Title}{Camera based 3D Dense Reconstruction for Mobile Robots}
\cvitem{Description}{
Designed an end to end pipeline for multi-view stereo dense 3D reconstruction from a handheld stereo-camera(\textbf{Intel RealSense}) that outputs stable dense pointclouds. In particular, extracted and tracked keyframe poses and keypoints from \href{https://vision.in.tum.de/research/vslam/basalt}{BASALT-VIO}. Encoded information from multiple camera views in a cost volume used for self-supervised training of a \textbf{U-Net} adapted architecture design-\href{https://arxiv.org/abs/2011.11814}{MonoRec}. Benchmarking of trajectory estimation done on rectified \href{https://arxiv.org/abs/1804.06120}{TUM-VI} dataset before deployment.}
% {Extracted keyframe poses and tracked keypoints in keyframes from VIO system - BASALT. Implemented a tool for visualization of extracted camera poses. Performed image recitification of TUM-VI dataset as images are captured from a fish-eye lens. Wrote a custom dataloader for EUROC type datasets for the dense reconstruction algorithm - MONOREC. Used a cost volume from multiple camera views for self-supervised training. Recorded stereo-images and IMU data from a Realsense Depth Camera and converted data to EUROC format. Synchronization of IMU data for optimal IMU tracking in VIO system for accurate camera pose estimation. Achieved accuracy of 5-10 cms on loop closure. Complete set-up of a pipeline for multi-view stereo 3D dense reconstruction from a handheld stereo-camera that outputs stable dense pointclouds.}
% \vspace{0.2 cm}

% \cvitem{Institution}{\textbf{Central Electronics Engineering Research Institute (CSIR-CEERI)}
% \hfill [May-Sept'21] 
% \hspace{5 em}
% \textit{Machine Learning Research Intern} \newline
% \scriptsize{\textit{Under \textbf{Dr. Madan Lakshmanan} (Senior Scientist, CEERI), \textbf{Dr. Sandeep Joshi} (Asst. Prof, BITS Pilani)}}
% }
% \cvitem{Project Title}{Subject State Classification System using PPG signals, (\textit{publication under review})}
% \cvitem{Description}{Classification of a person as fatigued or non-fatigued based on cardiac PPG signals of a human subject. Conducted data preprocessing, benchmarking of classification algorithms, along with various feature extraction strategies and regularisation techniques. Achieved state-of-the-art accuracy of 86\% with (TBD......) NN-based implementation.}

% Conducted intensive data preprocessing to refine the raw real-world data. Several heart rate and heart rate variability measures were extracted as features. Benchmarked various classification algorithms like Support Vector Machines, Random Forest, K Nearest Neighbors, Artificial Neural Network. For Neural network-based implementation, explored various feature extraction strategies and regularisati  on techniques. Empirically observed that NN-based implementation outperform other methods.
% \cvitem{Code}{}
% \newpage

\section{Technical Strengths}
\cvitem{Languages}{ Python, C++, C, JAVA, C\#, MATLAB, Latex, HTML, CSS}
\cvitem{Softwares}{Pytorch, Tensorflow, Keras, JAX, Numpy, OpenCV, Unity, Gazebo, Verilog, \newline Robot Operating System (ROS), AutoCAD, Android Studio}
% \cvitem{Platforms}{Windows, Linux(Ubuntu, PopOS)}

\section{Research Projects}

\cvitem{Title}{\textbf{Project Kratos, A Mars Rover} [\href{https://kratos-the-rover.github.io/}{Website}]   [\href{https://github.com/Kratos-The-Rover}{Code}]   [\href{https://www.youtube.com/watch?v=Ok03ddqfSL4&t=154s}{Demo Video}]\hfill [2020 - 2022]  \hspace{19 em}
\textit{Autonomous Subsystem lead}}
\cvitem{Description}{Development of a mars rover as part of the University Rover Challenge(\href{https://urc.marssociety.org/}{URC}). Team lead of the Autonomous Subsystem, responsible for autonomous traversal. Program design, implementation and deployment of mapping, planning and control nodes on Jetson Xavier for obstacle avoidance and object tracking(arrows, ARTags). 
\newline 
\textbf{Path planning and Perception}-Implemented A*, RRT*, Dijkstra’s on a 4-adjacency grid graph obtained from binary occupancy grid generated by ZED2i camera; 
% Simulation on Unity Game Engine and Gazebo. 
% \newline 
% \textbf{Perception}-Used ZED2i camera for generating binary occupancy grid of environment. 
\newline 
\textbf{Tracking}-Employed transfer learning on \textbf{YOLOv3}, \textbf{Mask R-CNN} algorithms for arrow detection. Achieved ROS integration using \href{https://github.com/leggedrobotics/darknet_ros}{darknet\_ROS}(20 fps). 
\newline 
\textbf{Control}-Wrote a custom P-controller based \textbf{visual servoing} algorithm for following arrows and ARTags. 
% \href{https://www.youtube.com/watch?v=Ok03ddqfSL4&t=154s}{[Demo Video]}
}
% \vspace{0.2 cm}
% https://drive.google.com/file/d/1olZWtBUl-I19Jh_E_7eZXGh8eZ5jlufS/view?usp=sharing

\vspace{0.2 cm}


\cvitem{Title}{\textbf{RGB Guided Sparse Depth Completion}
\hfill [Jun'21-Present]
\hspace{20 em}
\textit{Prof. Sravan Danda, Prof. Aditya Challa, BITS Goa}
}
\cvitem{Description}{Existing methods for \textbf{depth completion and estimation} tend to overfit with very less generalization across datasets. Focused on developing methods to identify statistical patterns in coupled RGB-depth maps. Redefined depth completion as interpolation problem on a grid graph with sparse-depth seed values. Using empirical results from hypotheses testing on LiDAR depth data for seed selection and context-aware \textbf{spatial seed propagation}. Achieved comparable results against computationally heavy deep learning based methods on \textbf{KITTI dataset}.}
% \cvitem{Description}{Devised statistical tests to validate hypotheses regarding depth data in outdoor(KITTI dataset) and indoor(NYUv2 depth) settings. Designing experiments for identifying statistical patterns in coupled RGB-depth maps. Exploring morphological approaches to generation of dense depth maps from sparse inputs. Working with CNN based architectures and supervised learning based approaches for real time depth map generation. Implemented an encoder-decoder model \textit{FastDepth} in pytorch \href{https://github.com/hardik01shah/SAiDL-Summer-Assignment-2021}{[code]}; working with Resnet, U-Net and MobileNet for encoding layers.}

% \cvitem{Title}{\textbf{Ensemble Learning for Activity Recognition}
% \hfill [Jan'22-May'22] 
% \hspace{20 em}
% \textit{Prof. Tanmay Verlekar, Prof. Ashwin Srinivasan, BITS}
% \newline
% \textit{APPCAIR Lab, BITS Goa in collaboration with Intel Labs}
% }
% \cvitem{Description}{Demonstrated use of ensemble learning for the task of activity recognition/video classification on the Something-Something-v2 dataset. Used a combination of 4 weak learners (TDN, TSN, TRN and TSM) to get final prediction. Weak learners were typically used for feature extraction. Explored various methods for combination of features, ultimately used for downstream classification. Achieved 62\% top-1 accuracy on the test set.}
\vspace{0.2 cm}

\cvitem{Title}{\textbf{Deep Hashing Networks for downstream image classification} {[\href{https://github.com/sushant1212/CS-F425-Project-CIMON}{Code}]}
\hfill [May '22] 
\hspace{20 em}
\textit{Prof. Tirtharaj Dash, BITS Goa}
}
\cvitem{Description}{Proposed the use of a deep hashing network (\href{https://arxiv.org/abs/2010.07804}{CIMON}) for image classification on the STL-10 dataset. Experimented with use of CIMON's rich hash codes as latent feature representations, traditionally used for efficient retrieval based tasks. Achieved comparable accuracy on the test set to existing methods in \textbf{unsupervised} setting.} 
% \vspace{0.2 cm}

% \cvitem{Title}{\textbf{Tetris Game Engine}
% \hfill [Jan-May '22] 
% \hspace{20 em}
% \textit{Spring '22 | CS F363: Compiler Construction Project}
% }
% \cvitem{Description}{Designed a complete end-to-end python based tetris game engine programming toolchain for creating interactive variants of tetris. Developed a new compiled programming language - tetris lang(tl) with syntax and semantics designed to aid a tetris game programmer. Wrote a custom SLY based compiler(lexer and parser) that uses user defined pattern action pairs and tokens to scan and parse input. \href{https://github.com/aryan02420/CS_F363-Compiler-Construction}{[Project Repository]}}
% \vspace{0.2 cm}

% \cvitem{Title}{\textbf{Self and Semi Supervised Learning}
% \hfill [May '22] 
% \hspace{20 em}
% \textit{Fall '21 | CS F425: Deep Learning Course Project}
% }
% \cvitem{Description}{Illustrated the contrastive learning based Barlow Twins method for self-supervised learning on the STL-10 dataset for image classification. Implemented pseudo-labelling technique for semi-supervised training of a Resnet-9. Benchmarked performance with traditional CNNs trained with supervision. \href{https://github.com/hrishi508/Self-and-Semi-Supervised-Learning}{[Project Repository]}}
% \vspace{0.2 cm}

% \cvitem{Title}{\textbf{Unity ROS Integration for simulation of differential drive robots}
% \hfill [Jan-May'21] 
% \hspace{5 em}
% \textit{Prof. Rakesh Warier, BITS} 
% % \textit{Prof. Rakesh Warier, BITS -} \scriptsize{\textit{Publication in progress.}}
% }  
% \cvitem{Description}{
% Designed a simulator for robot interaction in complex environments using Unity Game Engine and integrated with ROS. Benchmarked the performance of Unity and baseline simulator (Gazebo) in terms of accuracy, precision and physics. Set up SLAM(Simultaneous Localization and Mapping) and visualization of obstacles from Unity in RViZ; implemented Dijkstra's algorithm on the map for autonomous traversal of the bot. \newline
% Code: [\href{https://github.com/hardik01shah/slam-gmapping-unity-ros}{SLAM}], [\href{https://github.com/hardik01shah/Unity-ROS-Basic-Simulation}{UnityROS}], [\href{https://github.com/hardik01shah/Lidar-Simulation-Using-Unity-ROS}{LiDAR Simulation}]
% }
% \vspace{0.2 cm}


% \cvitem{Title}{\textbf{Project Kratos} [\href{https://kratosbitsgoa.com/}{Website}] \hfill [2020 - 2022]  \hspace{19 em}
% \textit{Autonomous Subsystem lead}}
% \cvitem{Description}{Development of a mars rover as part of the University Rover Challenge (URC). Team lead of the Autonomous Subsystem, responsible for autonomous traversal. Simulated autonomous 2D path planning for 3D bots with ROS using A*, RRT*, Dijkstra’s; testing on Unity Game Engine and Gazebo. Implemented YOLOv3, Mask RCNN algorithms, transfer learning for arrow detection and tested it on the real world using a robot to detect AR Tags and Arrows. Achieved ROS integration using darknet\_ROS. Complete implementation, deployment and testing of a mapping, planning and control based programming stack on the rover. [\href{https://github.com/Kratos-The-Rover}{Code}] }
% \vspace{0.2 cm}

% \cvitem{Title}{\textbf{Artpark Robotics Challenge} [\href{https://www.artpark.in/call-for-participation}{Website}]
% \hfill [2021]  
% }
% \vspace{-0.01 cm}
% % \cvitem{Supervisors}{Prof. Shibu Clement, BITS Pilani, Goa}
% \cvitem{Description}{Development of janitor bot, selected for stage 2: among top 28 teams out of 200+.
% }
% \vspace{0.2 cm}



\section{Relevant Coursework}
\cvitem{ETH Zurich*}{Probabilistic Aritificial Intelligence, Information Security, Computer Vision, Planning and Decision Making for Autonomous Robots, Vision Algorithms for Mobile Robotics \scriptsize{\textit{*in progress}}}
% \cvitem{}{}
\cvitem{BITS: CS}{Data Structures and Algorithms, Operating Systems, Computer Architecture, Database Systems, Compilers, Discrete Mathematical Structures in Computer Science}
\cvitem{BITS: ML}{Applied Statistical Methods, Foundations of Data Science, Machine Learning, Deep Learning, Artificial Intelligence}

% \section{Certifications}
% \cvlistitem{Introduction to Aviation and Aerodynamics - CTE Bits Goa}
% \cvlistitem{Version Control with Git by Atlassian - Coursera}
% \cvlistitem{Machine Learning by Stanford - Coursera}
% \cvlistitem{Deep Learning Specialization by DeepLearning.ai - Coursera}

\section{Leadership and Teaching}
\cventry{2021-2022}{Subsystem Lead, Autonomous Subsystem Project Kratos}{BITS Goa}{}{}{Managing a team of 14 members.  Continuous designing and improvement of all the framework components through research. Managed manufacturing, fabrication and integration of the essential rover components for the subsystem. Involved in close collaborations with other teams.}

\cventry{Fall 2022}{Student Mentor: ASCII Mentorship Programme}{BITS Goa}{\href{https://github.com/ASCII-Mentorships}{[Github Org]}}{}{Mentoring a team of 15 second and third year students in a semester long project towards exploring core domains in Computer Science. \href{https://github.com/ASCII-Mentorships/Sign-Language-Translator}{[Project Repository]}}
\cventry{Spring 2021}{Teaching Assistant: Discrete Structures in Computer Science}{BITS Goa}{}{}{}
% \cventry{Summer 2021}{Mentor: Autonomous Traversal for Differential Drive Robots}{}{[\href{https://github.com/Kratos-The-Rover/QSTP-2021---Autonomous-Subsystem}{Course Repo}]}{}
% {Mentored and taught 30 first year students path planning and control theory concepts for differential drive robots. Also covered simple deep learning techniques for perception.}

% \cventry{2020-2021}{Student Mentor, Institute Peer Mentorship Programme}{}{}{}{Responsible for mentoring a group of 7 freshmen to help adjust to the new environment, academically and socially and guide them towards a holistic development.}
% \cventry{2019-2021}{Volunteer, Abhigyaan}{}{[\href{abhigyaan-bpgc.in}{Website}]}{}{Abhigyaan is a social service initiative by the students of BITS-Goa. It aims at providing education to underprivileged children living in the BITS campus and nearby areas. I taught Physics and Chemistry to students of 11th and 12th grade, and helped them prepare for competitive exams.}

\section{Awards and Achievements}
\cvitem{2022}{\href{https://urc.marssociety.org/}{University Rover Challenge}, Utah: Project Kratos secured \textbf{1st} position in India}
\cvitem{2022}{\href{https://www.anatolianrover.space/}{Anatolian Rover Challenge}, Turkey: Project Kratos secured \textbf{2nd} position globally}
% \cventry{Spring 2022}{Runner Up - Postman API Hackathon 1.0}{}{}{}{Problem Statement was to create full-stack applications using APIs.  \href{https://powerful-waters-71400.herokuapp.com/}{[Submission]}}
\cvitem{2022}{Recipient of \textbf{DAAD WISE} research scholarship(Germany)}
% \cvitem{2022}{Recipient of \textbf{MITACS Globalink} research scholarship(Canada)}
% \cvitem{2022}{Recipient of the Singapore International Pre-Graduate Award (\textbf{SIPGA})}
% \cvitem{Summer 2021}{Selected for the \textbf{Microsoft Engage} Mentorship Program 2021}

% \section{Extra Curricular Activities}
% % \cvlistitem{Selected for the Microsoft Engage Mentorship Program 2021.}
% % \cvlistitem{Studied aerodynamic structures and design as part of Center for Technical Education.}
% % \cvlistitem{Involved in competitive programming, rated Div1 on CodeChef (1800+ rating).}
% \cvlistitem{Part of college tennis team, representing institute at state and national tournaments.}
% \cvlistitem{Completed the Goa River Half Marathon 2019 - 21km.}
% % \cvlistitem{Built RC planes and drones as part of the  Aerodynamics club BITSG}
% % \cvlistitem{Wrote articles on aviation as part of the Aerodynamics Club BITSG.}
% \cvlistitem{Interests: Reading, Cycling, Swimming, Trekking, Chess}

% \section{Languages}
% \cvitemwithcomment{Language 1}{Skill level}{Comment}
% \cvitemwithcomment{Language 2}{Skill level}{Comment}
% \cvitemwithcomment{Language 3}{Skill level}{Comment}

% \section{Computer skills}
% \cvdoubleitem{category 1}{XXX, YYY, ZZZ}{category 4}{XXX, YYY, ZZZ}
% \cvdoubleitem{category 2}{XXX, YYY, ZZZ}{category 5}{XXX, YYY, ZZZ}
% \cvdoubleitem{category 3}{XXX, YYY, ZZZ}{category 6}{XXX, YYY, ZZZ}






% \section{Extra 2}
% \cvlistdoubleitem{Item 1}{Item 4}
% \cvlistdoubleitem{Item 2}{Item 5\cite{book1}}
% \cvlistdoubleitem{Item 3}{Item 6. Like item 3 in the single column list before, this item is particularly long to wrap over several lines.}

% \section{References}
% \begin{cvcolumns}
%   \cvcolumn{Category 1}{\begin{itemize}\item Person 1\item Person 2\item Person 3\end{itemize}}
%   \cvcolumn{Category 2}{Amongst others:\begin{itemize}\item Person 1, and\item Person 2\end{itemize}(more upon request)}
%   \cvcolumn[0.5]{All the rest \& some more}{\textit{That} person, and \textbf{those} also (all available upon request).}
% \end{cvcolumns}

% % Publications from a BibTeX file without multibib
% %  for numerical labels: \renewcommand{\bibliographyitemlabel}{\@biblabel{\arabic{enumiv}}}% CONSIDER MERGING WITH PREAMBLE PART
% %  to redefine the heading string ("Publications"): \renewcommand{\refname}{Articles}
% \nocite{*}
% \bibliographystyle{plain}
% \bibliography{publications}                        % 'publications' is the name of a BibTeX file

% % Publications from a BibTeX file using the multibib package
% %\section{Publications}
% %\nocitebook{book1,book2}
% %\bibliographystylebook{plain}
% %\bibliographybook{publications}                   % 'publications' is the name of a BibTeX file
% %\nocitemisc{misc1,misc2,misc3}
% %\bibliographystylemisc{plain}
% %\bibliographymisc{publications}                   % 'publications' is the name of a BibTeX file

% \clearpage
% %-----       letter       ---------------------------------------------------------
% % recipient data
% \recipient{Company Recruitment team}{Company, Inc.\\123 somestreet\\some city}
% \date{January 01, 1984}
% \opening{Dear Sir or Madam,}
% \closing{Yours faithfully,}
% \enclosure[Attached]{curriculum vit\ae{}}          % use an optional argument to use a string other than "Enclosure", or redefine \enclname
% \makelettertitle

% Lorem ipsum dolor sit amet, consectetur adipiscing elit. Duis ullamcorper neque sit amet lectus facilisis sed luctus nisl iaculis. Vivamus at neque arcu, sed tempor quam. Curabitur pharetra tincidunt tincidunt. Morbi volutpat feugiat mauris, quis tempor neque vehicula volutpat. Duis tristique justo vel massa fermentum accumsan. Mauris ante elit, feugiat vestibulum tempor eget, eleifend ac ipsum. Donec scelerisque lobortis ipsum eu vestibulum. Pellentesque vel massa at felis accumsan rhoncus.

% Suspendisse commodo, massa eu congue tincidunt, elit mauris pellentesque orci, cursus tempor odio nisl euismod augue. Aliquam adipiscing nibh ut odio sodales et pulvinar tortor laoreet. Mauris a accumsan ligula. Class aptent taciti sociosqu ad litora torquent per conubia nostra, per inceptos himenaeos. Suspendisse vulputate sem vehicula ipsum varius nec tempus dui dapibus. Phasellus et est urna, ut auctor erat. Sed tincidunt odio id odio aliquam mattis. Donec sapien nulla, feugiat eget adipiscing sit amet, lacinia ut dolor. Phasellus tincidunt, leo a fringilla consectetur, felis diam aliquam urna, vitae aliquet lectus orci nec velit. Vivamus dapibus varius blandit.

% Duis sit amet magna ante, at sodales diam. Aenean consectetur porta risus et sagittis. Ut interdum, enim varius pellentesque tincidunt, magna libero sodales tortor, ut fermentum nunc metus a ante. Vivamus odio leo, tincidunt eu luctus ut, sollicitudin sit amet metus. Nunc sed orci lectus. Ut sodales magna sed velit volutpat sit amet pulvinar diam venenatis.

% Albert Einstein discovered that $e=mc^2$ in 1905.

% \[ e=\lim_{n \to \infty} \left(1+\frac{1}{n}\right)^n \]

% \makeletterclosing

%\clearpage\end{CJK*}                              % if you are typesetting your resume in Chinese using CJK; the \clearpage is required for fancyhdr to work correctly with CJK, though it kills the page numbering by making \lastpage undefined
\end{document}


%% end of file `template.tex'.
